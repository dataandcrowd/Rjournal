% !TeX root = RJwrapper.tex
\title{Comparison of Spatial Interpolation Methods for Air Pollution
Prediction\texttt{:} gstat and mgcv}
\author{by Hyesop Shin}

\maketitle

\abstract{%
Understanding the association between pollution exposure and the
deleterious effect on population health is a vital precursor for
modelling. However, there has been less comparisons of spatial
interpolations for pollution predictions due to different assumptions
and mathematical processes. In addition, SI has been provided in a
coarse temporal scale which is difficult to understand the dynamics of
exposure by mobility patterns and seasonal effects. This paper aims to
compare spatial interpolation methods to predict air pollution at a
finer temporal scale. 57 pollution stations around Seoul, S.Korea were
collected for the comparison between Universal Kriging and Generalised
Additive Model, with additional weights on road layers as an effect of
roadside pollution fields. Neither of the interpolation methods was
noticeably superior to the other, but the sparse station data meant that
only very smoothed large-scale fields could be recovered, which did not
accurately represent the extremes observed at individual stations..
}

\hypertarget{introduction}{%
\subsection{Introduction}\label{introduction}}

Population health has been seriously threatened by daily ambient air
pollution in South Korea. Despite the efforts to legislate national
pollution standards, daily peaks in recent winters and springs have
already exceeded the standards countless times. During 5th-8th March
2019, the entire country experienced over 200µg/m\textsuperscript{3} of
PM\textsubscript{10} and smog episodes; the national authorities warned
everyone to reduce outdoor activities. As most of Seoul's areas tend to
experience disastrous levels of pollution frequently, residents can be
exposed to air pollution unconsciously, which in the long-term can lead
to respiratory or cardiovascular ailments \citep{Zhang2013}. Thus,
understanding the spatial and temporal aspects of air pollution and its
relationship with exposure is crucial.

Exposure research has exploited spatial interpolation (SI) to
investigate the relationship between ambient air pollution and
population exposure in a spatial context -- ``which places have high air
pollution?'' and ``how many people can potentially become unwell from
high episodes?''. SI is a statistical method that can compute pollution
fields over a wide area with a given set of point measurements. SI can
mainly be split into a group that follows the assumption of spatial
autocorrelation (e.g.~Inverse Distance Weighted (IDW), Kriging), and a
group on statistical inference e.g.~generalised linear models and
generalised additive models \citep{Wood2019}. Methods that take spatial
autocorrelation into account assume that the values tend to be more
similar when closer together, whereas methods that use (spatial)
statistical inference delineate the inferential surfaces over a region
by minimising the residuals of the model.

However, when air pollution monitoring stations such as those in Seoul
are small in number compared to the size of the city, the estimation of
the potential population at risk due to air pollution can be completely
different depending on the measurement method used \citep{Wu2019}.
\citet{Wong2004} used four spatial interpolation methods -- spatial
averaging, nearest neighbour, IDW, and Kriging - to estimate children's
exposure to air pollution across the USA. The outcomes of the four
methods only showed a small difference of PM\textsubscript{10} and
O\textsubscript{3} where the monitoring stations were denser, for
example in the northeastern cities and urban California, but was
difficult to predict in the mountainous regions and high-altitude zones.
\citet{Aalto2013} compared a Generalised Additive Model (GAM) and an
Ordinary Kriging (OK) to predict monthly mean temperature and
precipitation and found that the GAM outperformed other methods by a
small amount but the biased distribution of stations (``concentrated in
the urban areas'' of Finland) might have evened out the RMSE measures.

In addition, previous studies have provided tentative estimates of
population risk based on annual or monthly statistics, but the
aggregated figure lacks the potential for including acute injuries after
an abrupt pollution rise, which may be more severe. However, there is
likely to be a greater difference when the population at risk is
measured at a finer temporal scale. This might support guidelines for
surveillance of short-term exposure.

This chapter aims to compare spatial interpolation methods that can
support estimating population exposure to daily air pollution in Seoul.
This study compared Universal kriging (UK), Generalised additive model
(GAM), UK with additional road effect, and GAM with additional road
effect to model PM\textsubscript{10}, and NO\textsubscript{2} in Seoul
as an intermediate phase of pollution modelling. Compared to previous
studies, this chapter generates outcomes on a 12-hour basis to
understand the daily cycle of the pollution over the city and
superimposes road effects to take into account small scale variables
that might be neglected in the typical spatial interpolation outcomes.

\hypertarget{geostatistics-and-statistical-inference-models}{%
\subsection{Geostatistics and Statistical Inference
Models}\label{geostatistics-and-statistical-inference-models}}

Recent literature of spatial statistics has shown advanced models such
as a bayesian fused spatio-temporal kriging or mathematically sound
equations to fit the domain contents. However, this article compares the
typical Kriging and GAM that appears on a two dimensional space.

\hypertarget{similarities}{%
\subsubsection{Similarities}\label{similarities}}

\hypertarget{the-distribution-of-the-data-should-follow-the-bell-curve}{%
\paragraph{The distribution of the data should follow the bell
curve}\label{the-distribution-of-the-data-should-follow-the-bell-curve}}

The robustness of data is only meaningful when the basic assumption that
the data is to follow the normal distribution is met. Hence, it is
always essential to explore the data.

\hypertarget{mean-and-the-variation-are-important}{%
\paragraph{Mean and the variation are
important}\label{mean-and-the-variation-are-important}}

Both Universal Kriging and GAM work with the mean value. In universal
kriging, it is important to find the average across the space

\hypertarget{differences}{%
\subsubsection{Differences}\label{differences}}

\begin{itemize}
\tightlist
\item
  Distance vs shape of the Spline
\item
  Fitting the Semivariogram versus Adjusting the Knots and Wiggliness
\item
  Computing Time
\end{itemize}

\hypertarget{empirical-experiment-using-gstat-and-mgcv}{%
\subsection{\texorpdfstring{Empirical Experiment using \texttt{gstat}
and
\texttt{mgcv}}{Empirical Experiment using gstat and mgcv}}\label{empirical-experiment-using-gstat-and-mgcv}}

\begin{itemize}
\tightlist
\item
  Using 57 stations surrounding Seoul, this article runs two SI models
\item
  We use NO\textsubscript{2} which chemically reacts well with
  O\textsubscript{3} thus is sensitive day and night, and
  PM\textsubscript{10} which is less sensitive to hourly differences but
  more sensitive to seasonal patterns
\item
\end{itemize}

\bibliography{RJreferences.bib}

\address{%
Hyesop Shin\\
MRC/CSO Social and Public Health Sciences Unit, University of Glasgow\\%
Berkeley Square, 99 Berkeley Street, Glasgow, G3 7HR\\
%
\url{https://www.gla.ac.uk/researchinstitutes/healthwellbeing/staff/hyesopshin/}\\%
%
\href{mailto:hyesop.shin@glasgow.ac.uk}{\nolinkurl{hyesop.shin@glasgow.ac.uk}}%
}
